\documentclass[10pt,letter]{article}
\usepackage{amsmath}
\usepackage{amssymb}
\usepackage{graphicx}
\usepackage{setspace}
\onehalfspacing
\usepackage{fullpage}
\newcommand{\R}{\mathbb{R}}	
\newcommand{\inner}{\langle\cdot,\cdot\rangle}
\newcommand{\inr}[2]{\langle #1, #2\rangle}
\newcommand\norm[1]{\left\lVert#1\right\rVert}

\begin{document}


\title{ACM104 Problem Set \#3 Solutions}

\author{Timur Kuzhagaliyev}

%\date{22nd October, 2017}
 
\maketitle 

\section*{Problem 1}

\paragraph{a)} It is possible to express the inner product using the norm:

\begin{align*}
\norm{u+v}^2 = \inr{u+v}{u+v} &= \inr{u}{u+v} + \inr{v}{u+v}
\\ & = \inr{u}{u} + \inr{u}{v} + \inr{v}{u} + \inr{v}{v}
\\ & = \inr{u}{u} + 2\cdot\inr{u}{v} + \inr{v}{v}
\\ & = \inr{u}{u} + 2\cdot\inr{u}{v} + \inr{v}{v}
\\ & = \norm{u}^2 + 2\cdot\inr{u}{v} + \norm{v}^2
\\ &\Downarrow
\\ \inr{u}{v} & = \frac{\norm{u+v}^2 - \norm{u}^2 - \norm{v}^2}{2}
\end{align*}

\paragraph{b)} There is only one inner product that can generate a norm. We can prove this by contradiction. Assume there are two distinct inner products $\inner_1$, $\inner_2$ that generate the same norm but are not identical. Then, pick two vectors $u$ and $v$ such that $\inr{u}{v}_1 \neq \inr{u}{v}_2$. By definition of $\inner_1$, $\inner_2$ and norm we have:
\begin{gather*}
\sqrt{\inr{v}{v}_1} = \sqrt{\inr{v}{v}_2} \qquad \textrm{for any } \; v \in V
\\
\Downarrow
\\
\inr{u+v}{u+v}_1 = \inr{u+v}{u+v}_2 \qquad \textrm{for any } \; u,v \in V
\end{gather*}
Note that, for any norm $\inner$, we have:
\begin{align*}
\inr{u+v}{u+v} &= \inr{u}{u+v} + \inr{v}{u+v}
\\ & = \inr{u}{u} + \inr{u}{v} + \inr{v}{u} + \inr{v}{v}
\\ & = \inr{u}{u} + 2\cdot\inr{u}{v} + \inr{v}{v}
\end{align*}
Apply this to $\inner_1$ and $\inner_2$:
\begin{align}
\inr{u+v}{u+v}_1 &= \inr{u}{u}_1 + 2\cdot\inr{u}{v}_1 + \inr{v}{v}_1
\\
\inr{u+v}{u+v}_2 &= \inr{u}{u}_2 + 2\cdot\inr{u}{v}_2 + \inr{v}{v}_2
\end{align}
Now subtract (2) from (1) and apply definition of $\inner_1$, $\inner_2$ and norm:
\begin{align*}
\inr{u+v}{u+v}_1 - \inr{u+v}{u+v}_2 &= \inr{u}{u}_1 - \inr{u}{u}_2 + 2\cdot\inr{u}{v}_1 - 2\cdot\inr{u}{v}_2 + \inr{v}{v}_1 - \inr{v}{v}_2
\\
0 &= 0 + 2\cdot\inr{u}{v}_1 - 2\cdot\inr{u}{v}_2 + 0
\\
2\cdot\inr{u}{v}_2 &= 2\cdot\inr{u}{v}_1
\\
\inr{u}{v}_2 &= \inr{u}{v}_1
\end{align*}
Hence $\inr{u}{v}_2 = \inr{u}{v}_1$, what contradicts our assumption and implies that both are the same inner product. Therefore, the inner product generating some norm must be unique.

\pagebreak

\section*{Problem 2}

\paragraph{a)} $\inr{f}{g}_1$ is not an inner product because it is not positive definite. Consider function $f(x) = 1$. Clearly, $f'(x) = 0$, which means $\inr{f}{f}_1 = \int^{1}_{0} f'(x)f'(x)\ dx = \int^{1}_{0} 0\ dx = 0$, but $f$ is not the zero vector.

$\inr{f}{g}_2$, on the other hand, is an inner product since it satisfies the properties of an inner product: it is bilinear, symmetric and positive-definite.

\paragraph{b)} The inner product is:

\begin{gather*}
\inr{f}{g} = \int^{1}_{0} (f(x)g(x) + f'(x)g'(x))\ dx
\end{gather*}

Cauchy-Schwarz inequality:

\begin{gather*}
\sqrt{\int^{1}_{0} (f(x)g(x) + f'(x)g'(x))\ dx} \leq \sqrt{\int^{1}_{0} (f(x)^2 + f'(x)^2)\ dx} \cdot \sqrt{\int^{1}_{0} (g(x)^2 + g'(x)^2)\ dx} 
\end{gather*}

Triangle inequality:

\begin{gather*}
\sqrt{\int^{1}_{0} ((f(x)+g(x))^2 + (f'(x)+g'(x))^2)\ dx} \leq \sqrt{\int^{1}_{0} (f(x)^2 + f'(x)^2)\ dx} + \sqrt{\int^{1}_{0} (g(x)^2 + g'(x)^2)\ dx} 
\end{gather*}

\paragraph{c)} Starting from $\textrm{cos}\,\theta$:

\begin{align*}
\textrm{cos}\,\theta &= \frac{\int^{1}_{0} (f(x)g(x) + f'(x)g'(x))\ dx}{\sqrt{\int^{1}_{0} (f(x)^2 + f'(x)^2)\ dx} \cdot \sqrt{\int^{1}_{0} (g(x)^2 + g'(x)^2)\ dx} }
\\ &= \frac{\int^{1}_{0} (e^x + 0)\ dx}{\sqrt{\int^{1}_{0} (1^2 + 0^2)\ dx} \cdot \sqrt{\int^{1}_{0} (e^{2x} + e^{2x})\ dx} }
\\ &= \frac{[e^x]^{1}_{0}}{\sqrt{[x]^{1}_{0}} \cdot \sqrt{[e^{2x}]^{1}_{0}} }
\\ &= \frac{e - 1}{\sqrt{1} \cdot \sqrt{e^{2} - 1} }
\\ &= \frac{e - 1}{\sqrt{(e - 1)(e + 1)}}
\\ &= \frac{\sqrt{(e - 1)}}{\sqrt{(e + 1)}}
\\ \theta &\approx 0.8233 \ \textrm{rad} \; \textrm{(4 d.p.)}
\end{align*}

\pagebreak

\section*{Problem 3}

See attached \texttt{.png} plots and Matlab files. For part (d), the minimum $p$ value achieved on the last iteration is $1014.6506$.

\section*{Problem 4}

\paragraph{a)} Finding the Gram matrix G using $L^2 = \int^1_0 f(x)g(x)\; dx$ inner product:

\begin{align*}
G &=
\left[ {\begin{array}{ccccc}
 \inr{1}{1} & \inr{1}{e^x} & \inr{1}{e^{2x}}  \\
 \inr{e^x}{1} & \inr{e^x}{e^x} & \inr{e^x}{e^{2x}}  \\
 \inr{e^{2x}}{1} & \inr{e^{2x}}{e^x} & \inr{e^{2x}}{e^{2x}}  \\
\end{array} } \right]
\\ &=
\left[ {\begin{array}{ccccc}
 \int^1_0 1\; dx & \int^1_0 e^x\; dx & \int^1_0 e^{2x}\; dx  \\
 \int^1_0 e^x\; dx & \int^1_0 e^{2x}\; dx & \int^1_0 e^{3x}\; dx  \\
 \int^1_0 e^{2x}\; dx & \int^1_0 e^{3x}\; dx & \int^1_0 e^{4x}\; dx  \\
\end{array} } \right]
\\ &=
\left[ {\begin{array}{ccccc}
 1 & e - 1 & \frac{1}{2}(e^2 - 1)  \\
 e - 1 & \frac{1}{2}(e^2 - 1) & \frac{1}{3}(e^3 - 1)  \\
 \frac{1}{2}(e^2 - 1) & \frac{1}{3}(e^3 - 1) & \frac{1}{4}(e^4 - 1)  \\
\end{array} } \right]
\end{align*}

\paragraph{b)} Note that $1$, $e^x$ and $e^{2x}$ are linearly independent since you cannot generate any one function by combining the others using only scalar coefficients. This implies that Gram matrix $G$ must be positive-definite.

\paragraph{c)} Using the inner product from Problem 2 results in the matrix $G_2$ seen below. Note that the matrix itself doesn't matter when it comes to positive-definiteness of $G_2$: since we're using the same vectors and these vectors are independent, the resultant Gram matrix would always be positive-definite.

\begin{align*}
G_2 &=
\left[ {\begin{array}{ccccc}
 \inr{1}{1}_2 & \inr{1}{e^x}_2 & \inr{1}{e^{2x}}_2  \\
 \inr{e^x}{1}_2 & \inr{e^x}{e^x}_2 & \inr{e^x}{e^{2x}}_2  \\
 \inr{e^{2x}}{1}_2 & \inr{e^{2x}}{e^x}_2 & \inr{e^{2x}}{e^{2x}}_2  \\
\end{array} } \right]
\\ &=
\left[ {\begin{array}{ccccc}
 \int^1_0 1\; dx             & \int^1_0 2\cdot e^x\; dx    & \int^1_0 3\cdot e^{2x}\; dx  \\
 \int^1_0 2\cdot e^x\; dx    & \int^1_0 2\cdot e^{2x}\; dx & \int^1_0 3\cdot e^{3x}\; dx  \\
 \int^1_0 3\cdot e^{2x}\; dx & \int^1_0 3\cdot e^{3x}\; dx & \int^1_0 5\cdot e^{4x}\; dx  \\
\end{array} } \right]
\\ &=
\left[ {\begin{array}{ccccc}
 1     &  2\,e - 2 & \frac{3}{2}(e^2 - 1) \\
 2\,e - 2 & e^2 - 1 & e^3 - 1  \\
 \frac{3}{2}(e^2 - 1) & e^3 - 1 & \frac{5}{4}(e^4 - 1)  \\
\end{array} } \right]
\end{align*}

\paragraph{d)} Repeating the point above: Since $1$, $e^x$ and $e^{2x}$ are linearly independent, the Gram matrix generated using any inner product for that vector space would be positive-definite. Therefore it's impossible to find $\inner_1$ and $\inner_2$ such that one generates a positive-definite Gram matrix and the other doesn't.

\section*{Problem 5}

See attached Matlab script. My calculation revealed that Ted Cruz's Wikipedia page is the most similar to US Constitution.

\end{document}