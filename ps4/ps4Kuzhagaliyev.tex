\documentclass[10pt,letter]{article}
\usepackage{amsmath}
\usepackage{amssymb}
\usepackage{graphicx}
\usepackage{setspace}
\onehalfspacing
\usepackage{fullpage}
\newcommand{\R}{\mathbb{R}}	
\newcommand{\inner}{\langle\cdot,\cdot\rangle}
\newcommand{\inr}[2]{\langle #1, #2\rangle}
\newcommand\norm[1]{\left\lVert#1\right\rVert}

\begin{document}


\title{ACM104 Problem Set \#4 Solutions}

\author{Timur Kuzhagaliyev}

%\date{3rd November, 2017}
 
\maketitle 

\section*{Problem 1}

\begin{align*}
A &=
\left[ {\begin{array}{ccc}
 1 &  2 & -1 \\
 0 &  -2 & 3 \\
 1 &  5 & -1 \\
 -3 &  1 & 1 \\
\end{array} } \right]
\qquad \textrm{and} \qquad
b = 
\left[ {\begin{array}{c}
 0 \\
 5 \\
 6 \\
 8 \\
\end{array} } \right]
\end{align*}
\begin{align*}
p(x) = \norm{Ax-b}^2 &= x^T (A^T A) x - 2\ x^T (A^T b) + \norm{b}^2
\\ &= x^T K x - 2 x f + c 
\end{align*}
\begin{align*}
K = A^T \cdot A &=
\left[ {\begin{array}{cccc}
 1 &  0 & 1 & -3 \\
 2 &  -2 & 5 & 1 \\
 -1 &  3 & -1 & 1 \\
\end{array} } \right]
\left[ {\begin{array}{ccc}
 1 &  2 & -1 \\
 0 &  -2 & 3 \\
 1 &  5 & -1 \\
 -3 &  1 & 1 \\
\end{array} } \right]
=
\left[ {\begin{array}{ccc}
 11 & 4  & -5 \\
 4 & 34  & -12 \\
 -5 & -12  & 12 \\
\end{array} } \right]
\\\\ f = A^T b &= 
\left[ {\begin{array}{ccc}
 1 &  2 & -1 \\
 0 &  -2 & 3 \\
 1 &  5 & -1 \\
 -3 &  1 & 1 \\
\end{array} } \right]
\left[ {\begin{array}{c}
 0 \\
 5 \\
 6 \\
 8 \\
\end{array} } \right]
=
\left[ {\begin{array}{c}
 -18 \\
 28 \\
 17 \\
\end{array} } \right]
\end{align*}
First we need to check if the matrix $K$ is positive definite. We know that matrix is positive definite if all of its principal minors are positive. Indeed, $\textrm{det}\,A_1 = 11 > 0$; $\textrm{det}\,A_2 = 11 \cdot 34 - 4^2 = 18 > 0$; $\textrm{det}\,A_3 = 11 \cdot 264 - 4 \cdot (-12) - 5 \cdot 122 = 2904 + 48 - 610 = 2342 > 0$; Therefore $K$ is positive definite and there exists a global minimizer $x^* = K^{-1}f$.

\pagebreak

\begin{align*}
x^* = K^{-1} f = K f &=
\left[ {\begin{array}{ccc}
 11 & 4  & -5 \\
 4 & 34  & -12 \\
 -5 & -12  & 12 \\
\end{array} } \right]
\left[ {\begin{array}{c}
 -18 \\
 28 \\
 17 \\
\end{array} } \right]
\\ &= \frac{1}{2342}
\left[ {\begin{array}{ccc}
 264 & 12 & 122 \\
 12 & 107 & 112 \\
 122 & 112  & 358 \\
\end{array} } \right]
\left[ {\begin{array}{c}
 -18 \\
 28 \\
 17 \\
\end{array} } \right]
\\ &=
\left[ {\begin{array}{c}
 -1 \\
 2 \\
 3 \\
\end{array} } \right]
\quad \textrm{(solution)}
\\ \textrm{LSE} = \sqrt{\norm{b}^2 - b^T A x^*} &= \sqrt{\sqrt{125}^2 - 125}
\\ &= 0\\
\end{align*}

The least squares error is $0$, so in this case $x^*$ is the exact solution.

\section*{Problem 2}

Note that in both $a)$ and $b)$ the degree of $p_n(x)$ is 1. We can derive $p_1(x)$ beforehand:

\begin{align*}
p_1(x) = f(x_o) \cdot L_1(x) + f(x_1) \cdot L_2(x) &= f(x_0) \cdot \frac{x - x_1}{x_0 - x_1} + f(x_1) \cdot \frac{x - x_0}{x_1 - x_0}
\\\\
&= \frac{f(x_0)}{x_0 - x_1} x - \frac{x_1 f(x_0)}{x_0 - x_1}
+ \frac{f(x_1)}{x_1 - x_0} x - \frac{x_0 f(x_1)}{x_1 - x_0}
\\\\
&= \frac{f(x_0) - f(x_1)}{x_0 - x_1} x + \frac{x_0 f(x_1) - x_1 f(x_0)}{x_0 - x_1}
\end{align*}

\paragraph{a)} Substituting $x_0 = a$ and $x_1 = b$ into $p_1(x)$:
\begin{align*}
p_1(x) &= \frac{f(a) - f(b)}{a - b} x + \frac{a f(b) - b f(a)}{a - b}
\end{align*}
\begin{align*}
\int_a^b f(x)\; dx \approx \int_a^b p_1(x)\; dx &= \int_a^b \left( \frac{f(a) - f(b)}{a - b} x + \frac{a f(b) - b f(a)}{a - b} \right) \; dx
\\
&= \left[ \frac{f(a) - f(b)}{a - b} \cdot \frac{x^2}{2} + \frac{a f(b) - b f(a)}{a - b} \cdot x \right]_a^b
\\
&= \frac{f(a) - f(b)}{a - b} \cdot \frac{b^2}{2} + \frac{a f(b) - b f(a)}{a - b} \cdot b
- \frac{f(a) - f(b)}{a - b} \cdot \frac{a^2}{2} - \frac{a f(b) - b f(a)}{a - b} \cdot a
\\
&= \frac{f(a) - f(b)}{a - b} \cdot \frac{b^2 - a^2}{2} - a f(b) + b f(a)
\\
&= \frac{1}{2} \cdot (f(b) - f(a))(b + a) - a f(b) + b f(a)
\\
&= \frac{1}{2} \left[ bf(b) + af(b) - b f(a) - a f(a) - 2 a f(b) + 2 b f(a) \right]
\\
&= \frac{1}{2} \left[ bf(b) - af(b) + b f(a) - a f(a) \right]
\\
&= \frac{1}{2} \left[ f(b) + f(a) \right]( b - a) \qquad \textrm{(trapezoid rule)}
\end{align*}

\paragraph{b)} Substituting $x_0 = \frac{1}{3}(a+b)$ and $x_1 = \frac{2}{3}(a+b)$ into $p_1(x)$:
\begin{align*}
p_1(x) &= \frac{f(x_0) - f(x_1)}{\frac{1}{3}(a + b) - \frac{2}{3}(a + b)} x + \frac{\frac{1}{3}(a + b) f(x_1) - \frac{2}{3}(a + b) f(x_0)}{\frac{1}{3}(a + b) - \frac{2}{3}(a + b)}
\\
&= \left[ f(x_1) - f(x_0) \right] \cdot \frac{3x}{a+b} + 2f(x_0) - f(x_1)
\end{align*}

I didn't replace $x_i$ in $f(x_i)$ to make the notation a bit more compact. I will substitute the actual value in the end.
\begin{align*}
\int_a^b f(x)\; dx \approx \int_a^b p_1(x)\; dx &= \int_a^b \left( \left[ f(x_1) - f(x_0) \right] \cdot \frac{3x}{a+b} + 2f(x_0) - f(x_1) \right) \; dx
\\
&= \left[ \frac{3}{2} \left[ f(x_1) - f(x_0) \right] \cdot \frac{x^2}{a+b} + 2xf(x_0) - xf(x_1) \right]_a^b
\\
&= \frac{3}{2} \left[ f(x_1) - f(x_0) \right] \cdot \frac{b^2}{a+b} + 2bf(x_0) - bf(x_1)
- \frac{3}{2} \left[ f(x_1) - f(x_0) \right] \cdot \frac{a^2}{a+b} - 2af(x_0) + af(x_1)
\\
&= \frac{3}{2} \left[ f(x_1) - f(x_0) \right] (b - a) + \left[ 2f(x_0) - f(x_1) \right] (b - a)
\\
&= \left[ \frac{3}{2} \cdot f(x_1) - \frac{3}{2} \cdot f(x_0) + 2f(x_0) - f(x_1) \right] (b - a)
\\
&= \left[ \frac{1}{2} f(x_1) - \frac{1}{2} f(x_0) \right] (b - a)
\\
&= \frac{1}{2} \left[ f(\frac{2}{3}(a+b)) + f(\frac{1}{3}(a+b)) \right] (b - a) \qquad \textrm{(open rule)}
\end{align*}

\paragraph{c)} Testing the approximations:

\begin{align*}
\int_0^1 e^x\; dx & \quad = e - 1 & & = 1.718281\hdots
\\
& \quad \approx \frac{1}{2}(e + 1) & & = 1.859140\hdots \quad \textrm{(trapezium rule)}
\\
& \quad \approx \frac{1}{2}(e^{\frac{2}{3}} + e^{\frac{1}{3}}) & & = 1.671673\hdots \quad \textrm{(open rule)}
\end{align*}

\begin{align*}
\int_0^\pi sin(x)\; dx & \quad = 1 - (-1) & & = 2
\\
& \quad \approx \frac{\pi}{2} & & = 1.570796\hdots \quad \textrm{(trapezium rule)}
\\
& \quad \approx \frac{\pi}{2}(\frac{\sqrt{3}}{2} + \frac{\sqrt{3}}{2}) & & = 2.720699\hdots \quad \textrm{(open rule)}
\end{align*}

For $e^x$, open rule has a smaller error than trapezoid rule, but trapezoid rule performs better for $sin(x)$.

\section*{Problem 3}
\begin{gather*}
y = f(x_1, x_2) = \beta_0 + \beta_1 x_1 + \beta_2 x_2 + \beta_3 x_1 x_2
\quad \textrm{and} \quad
\beta^* = (\beta^*_0, \beta^*_1, \beta^*_2, \beta^*_3)^T
\end{gather*}

\paragraph{a)} Deriving a system of normal equations on $\beta^*$:

\begin{equation*}
\begin{aligned}[c]
r_i &= y^{(i)} - f(x_1^{(i)}, x_2^{(i)})
\\ &= y^{(i)} - 
\left[ {\begin{array}{cccc}
 1 & x_1^{(i)} & x_2^{(i)} & x_1^{(i)} x_2^{(i)} \\
\end{array} } \right]
\left[ {\begin{array}{c}
\beta^*_0 \\
\beta^*_1 \\
\beta^*_2 \\
\beta^*_3 \\
\end{array} } \right]
\end{aligned}
\qquad\Rightarrow\qquad
\begin{aligned}[c]
r &=  
\left[ {\begin{array}{c}
y^{(1)} \\
y^{(2)} \\
\vdots \\
\vdots \\
y^{(m)} \\
\end{array} } \right]
-
\left[ {\begin{array}{cccc}
 1 & x_1^{(1)} & x_2^{(1)} & x_1^{(1)} x_2^{(1)} \\
 1 & x_1^{(2)} & x_2^{(2)} & x_1^{(2)} x_2^{(2)} \\
 \vdots & \vdots & \vdots & \vdots \\
 \vdots & \vdots & \vdots & \vdots \\
 1 & x_1^{(m)} & x_2^{(m)} & x_1^{(m)} x_2^{(m)} \\
\end{array} } \right]
\left[ {\begin{array}{c}
\beta^*_0 \\
\beta^*_1 \\
\beta^*_2 \\
\beta^*_3 \\
\end{array} } \right]
\end{aligned}
\end{equation*}

See attached \texttt{ps4problem3Kuzhagaliyev.m} file and plots for other solutions.

\section*{Problem 4}

See attached \texttt{ps4problem4Kuzhagaliyev.m} file and plot for solutions.

\section*{Problem 5}

See attached \texttt{ps4problem5Kuzhagaliyev.m} and plot for solution.

\end{document}